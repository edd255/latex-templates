%-------------------------------------------------------------------------------
% Original Template: Friedrich Günther and Alexander Rogovskyy
% Modifications by: abaddon
% Source: https://gitlab.cs.uni-saarland.de/fsr/latextemplate
%-------------------------------------------------------------------------------

% The class provides the "article"-like element of the koma-script collection.
% The document layout of the class is less ‘strident’ than that of article, and
% it offers much more flexibility than article via other elements of the
% koma-script collection.
%
% https://www.ctan.org/pkg/scrartcl
\documentclass[12pt,
               twocolumn=false,
               twoside=false,
               DIV=calc
]{scrartcl}

%----- Packages ----------------------------------------------------------------

%%---- Fonts -------------------------------------------------------------------

% The package allows the user to select font encodings, and for each encoding
% provides an interface to ‘font-encoding-specific’ commands for each font. Its
% most powerful effect is to enable hyphenation to operate on texts containing
% any character in the font.
%
% https://www.ctan.org/pkg/fontenc
\usepackage[T1]{fontenc}

% The Latin Modern family of fonts consists of 72 text fonts and 20 mathematics
% fonts, and is based on the Computer Modern fonts released into public domain
% by AMS (Copyright 1997 AMS). The lm font set contains a lot of additional
% characters, mainly accented ones, but not exclusively.
%
% https://www.ctan.org/pkg/lm
\usepackage{lmodern}

% These fonts are available under the IBM/MIT X Consortium Courier Typefont
% agreement. The distribution contains PFA outline fonts (ASCII-encoded Type 1),
% and AFM files.
%
% https://www.ctan.org/pkg/courier
\usepackage{courier}

% This font is useful for typesetting the mathematical symbols for the natural
% numbers (N), whole numbers (Z), rational numbers (Q), real numbers (R),
% complex numbers (C), and a couple of others which are sometimes needed.
%
% https://www.ctan.org/tex-archive/fonts/doublestroke
\usepackage{dsfont}


%%---- Encodings ---------------------------------------------------------------

% The package translates various standard and other input encodings into a ‘LaTeX
% internal language’. The internal language is expressed entirely in TeX’s base
% encoding (standard ASCII printable characters, carriage control tokens and TeX
% control sequences, the latter mostly defined by LaTeX).
%
% https://www.ctan.org/pkg/inputenc
\usepackage[utf8]{inputenc}


%%---- Languages ---------------------------------------------------------------

% This package manages culturally-determined typographical (and other) rules
% for a wide range of languages. A document may select a single language to be
% supported, or it may select several, in which case the document may switch
% from one language to another in a variety of ways.
%
% https://www.ctan.org/pkg/babel
\usepackage[english]{babel}

% This package provides advanced facilities for inline and display quotations.
% It is designed for a wide range of tasks ranging from the most simple
% applications to the more complex demands of formal quotations. The facilities
% include commands, environments, and user-definable ‘smart quotes’ which
% dynamically adjust to their context. Quotation marks are switched
% automatically if quotations are nested and they can be adjusted to the current
% language if the babel package is available. There are additional facilities
% designed to cope with the more specific demands of academic writing,
% especially in the humanities and the social sciences. All quote styles as well
% as the optional active quotes are freely configurable.
%
% https://www.ctan.org/pkg/csquotes
\usepackage[style=english]{csquotes}


%%---- URLs --------------------------------------------------------------------

% The command \url is a form of verbatim command that allows linebreaks at
% certain characters or combinations of characters, accepts reconfiguration, and
% can usually be used in the argument to another command.
%
% https://www.ctan.org/pkg/url
\usepackage{url}

% The hyperref package is used to handle cross-referencing commands in LaTeX to
% produce hypertext links in the document.
%
% https://www.ctan.org/pkg/hyperref
\usepackage[
    colorlinks=true,
    linktoc=all,
    linkcolor=black,
    urlcolor=black
]{hyperref}


%%---- Citations ---------------------------------------------------------------

% The bundle provides a package that implements both author-year and numbered
% references, as well as much detailed of support for other bibliography use.
% Also Provided are versions of the standard BibTeX styles that are compatible
% with natbib—plainnat, unsrtnat, abbrnat. The bibliography styles produced by
% custom-bib are designed from the start to be compatible with natbib.
%
% https://www.ctan.org/pkg/natbib
\usepackage{natbib}


%%---- Colors and Graphics -----------------------------------------------------

% The package starts from the basic facilities of the color package, and
% provides easy driver-independent access to several kinds of color tints,
% shades, tones, and mixes of arbitrary colors. It allows a user to select a
% document-wide target color model and offers complete tools for conversion
% between eight color models.
%
% https://www.ctan.org/pkg/xcolor
\usepackage[svgnames]{xcolor}

% The package builds upon the graphics package, providing a key-value interface
% for optional arguments to the \includegraphics command. This interface
% provides facilities that go far beyond what the graphics package offers on its
% own.
%
% https://www.ctan.org/pkg/graphicx
\usepackage{graphicx}
\graphicspath{ {.} }

% A package built on the standard LaTeX graphics package to perform all the
% different sorts of rotation one might like, including complete figures and
% tables with their captions.
%
% If you want continuous text (i.e., more than one page) set in landscape mode,
% use the lscape package instead. The rotating packages only deals in rotated
% boxes (or floats, which are themselves boxes), and boxes always stay on one
% page.
%
% If you need to use the facilities of the float in the same document, load
% rotating.sty via rotfloat, which smooths the path between the rotating and
% float packages.
%
% The package is now part of the latex-graphics bundle, which is one of the
% collections in the LaTeX ‘required’ set of packages.
%
% https://www.ctan.org/pkg/rotating
\usepackage{rotating}


%%---- Typography---------------------------------------------------------------

% The package provides an easy and flexible user interface to customize page
% layout, implementing auto-centering and auto-balancing mechanisms so that the
% users have only to give the least description for the page layout. For
% example, if you want to set each margin 2cm without header space, what you
% need is just \usepackage[margin=2cm,nohead]{geometry}.
%
% The package knows about all the standard paper sizes, so that the user need
% not know what the nominal ‘real’ dimensions of the paper are, just its
% standard name (such as a4, letter, etc.).
%
% An important feature is the package’s ability to communicate the paper size
% it’s set up to the output (whether via DVI \specials or via direct interaction
% with pdf(La)TeX).
%
% https://www.ctan.org/pkg/geometry
% \usepackage[
%         a4paper,
%         lmargin={2.5cm},
%         rmargin={2.5cm},
%         tmargin={2.5cm},
%         bmargin={2.5cm}
% ]{geometry}

% Provides support for setting the spacing between lines in a document. Package
% options include singlespacing, onehalfspacing, and doublespacing.
% Alternatively the spacing can be changed as required with the \singlespacing,
% \onehalfspacing, and \doublespacing commands. Other size spacings also
% available.
%
% https://www.ctan.org/pkg/setspace
\usepackage[onehalfspacing]{setspace}

% Change of default bold font
\renewcommand{\bfdefault}{b}

% This package is the part of the Koma-Script bundle that provides an end user
% interface to scrlayer, allowing the user to define and manage page styles by
% controlling page headers and footers. The end user interface is compatible
% enough to that of the earlier scrpage2 package, so that users should easily be
% able to switch from the older package to the newer, more powerful, one.
%
% https://www.ctan.org/pkg/scrlayer-scrpage
\usepackage{scrlayer-scrpage}

% Set page style
\pagestyle{scrheadings}

% Überschriften in fett und mit Serifen
\setkomafont{disposition}{\bfseries}

% Improves the interface for defining floating objects such as figures and
% tables. Introduces the boxed float, the ruled float and the plaintop float.
% You can define your own floats and improve the behaviour of the old ones.
%
% The package also provides the H float modifier option of the obsolete here
% package. You can select this as automatic default with
% \floatplacement{figure}{H}.
%
% https://www.ctan.org/pkg/float
\usepackage{float}

% A paragraph-opening line that appears by itself at the bottom of a page or
% column, thus separated from the rest of the text. Mnemonically, an orphan is
% "alone at the bottom" (of the family tree but, in this case, of the page).
% (https://en.wikipedia.org/wiki/Widows_and_orphans) Maximize the penalty for
% it.
\clubpenalty 10000

% A paragraph-ending line that falls at the beginning of the following page or
% column, thus separated from the rest of the text. Mnemonically, a widow is
% "alone at the top" (of the family tree but, in this case, of the page)
% (https://en.wikipedia.org/wiki/Widows_and_orphans). Maximize the penalty for
% it.
\widowpenalty 10000


%%---- Miscelleanous ------------------------------------------------------------

% This package provides user control over the layout of the three basic list
% environments: enumerate, itemize and description. It supersedes both enumerate
% and mdwlist (providing well-structured replacements for all their
% funtionality), and in addition provides functions to compute the layout of
% labels, and to ‘clone’ the standard environments, to create new environments
% with counters of their own.
%
% https://www.ctan.org/pkg/enumitem
\usepackage{enumitem}

% Default settings for enumerations: same indent, no additional spacing between
% points
\setlist{listparindent=\parindent, parsep=0pt}


%%---- Computer science ---------------------------------------------------------
%%%--- TikZ --------------------------------------------------------------------

% PGF is a macro package for creating graphics. It is platform- and
% format-independent and works together with the most important TeX backend
% drivers, including pdfTeX and dvips. It comes with a user-friendly syntax
% layer called TikZ.
%
% https://www.ctan.org/pkg/pgf
\usepackage{tikz}

% The package provides a macro for drawing trees with TikZ using the easy syntax
% of Alexis Dimitriadis’ Qtree. It improves on TikZ’s standard tree-drawing
% facility by laying out tree nodes without collisions; it improves on Qtree by
% adding lots of features from TikZ (for example, edge labels, arrows between
% nodes); and it improves on pst-qtree in being usable with pdfTeX and XeTeX.
%
% https://www.ctan.org/pkg/tikz-qtree
\usepackage{tikz-qtree}

% The libraries arrows and arrows.spaced from older versions of pgf are still
% available for compatibility, but they are considered deprecated.
% The standard arrow tips, which are loaded by the library arrows.meta, are
% documented in Section 16.5.
%
% https://tikz.dev/library-arrows
\usetikzlibrary{arrows}

% The automata (drawing) library is intended to make it easy to draw finite
% automata and Turing machines. It does not cover every situation imaginable,
% but most finite automata and Turing machines found in text books can be drawn
% in a nice and convenient fashion using this library.
%
% https://tikz.dev/library-automata
\usetikzlibrary{automata}

% A tiny library that make the interaction with the babel package easier.
% Despite the name, it may also be useful in other contexts, namely whenever the
% catcodes of important symbols are changed globally. Normally, using this
% library is always a good idea; it is not always loaded by default since in
% some rare cases it may break old code.
%
% https://tikz.dev/library-babel
\usetikzlibrary{babel}

% The general-purpose drawing package TiKZ can be used to typeset commutative
% diagrams and other kinds of mathematical pictures, generating high-quality
% results. The purpose of this package is to make the process of creation of
% such diagrams easier by providing a convenient set of macros and reasonable
% default settings. This package also includes an arrow tip library that match
% closely the arrows present in the Computer Modern typeface.
\usetikzlibrary{cd}

% The circuit libraries can be used to draw different kinds of electrical or
% logical circuits. There is not a single library for this, but a whole
% hierarchy of libraries that work in concert. The main design goal was to
% create a balance between ease-of-use and ease-of-extending, while creating
% high-quality graphical representations of circuits.
%
% https://tikz.dev/library-circuits
\usetikzlibrary{circuits}

% The positioning library allows to place nodes at a specified direction and
% distance from other nodes.
\usetikzlibrary{positioning}

% Configuration of automata
\tikzset{
    ->,
    >=stealth,
    node distance=3cm,
    every state/.style={thick, fill=gray!10},
    initial text=$ $,
}


%%---- Algorithms --------------------------------------------------------------

% The package provides several commands from the IEEEtran class file so that
% they can be used under other LaTeX classes; the package should not be used in
% a document that uses that parent class. The user guide only covers the ways in
% which usage differs from that in the parent class; thus the documentation of
% that class should be regarded as part of the documentation for the package.
%
% https://www.ctan.org/pkg/ieeetrantools
\usepackage{IEEEtrantools}

% This package provides many possibilities to customize the layout of
% algorithms. You can use one of the predefined layouts (pseudocode, pascal, C,
% and others), with or without modifications, or you can define a completely new
% layout for your specific needs.
%
% https://tug.ctan.org/macros/latex/contrib/algorithmicx/algorithmicx.pdf
\usepackage[noend]{algpseudocode}
\usepackage{algorithm}

% Renew function to class (use procedure instead for functions)
\algrenewcommand\algorithmicfunction{\textbf{class}}


%%---- Maths -------------------------------------------------------------------

% This package provides tons of math symbols, like arrows, operators, special
% characters, geometric figures etc.
%
% https://ctan.math.washington.edu/tex-archive/fonts/amsfonts/doc/amssymb.pdf
\usepackage{amssymb}

% The package facilitates the kind of theorem setup typically needed in American
% Mathematical Society publications. The package offers the theorem setup of the
% AMS document classes (amsart, amsbook, etc.) encapsulated in LaTeX package
% form so that it can be used with other document classes.
%
% https://www.ctan.org/pkg/amsthm
\usepackage{amsthm}

% The principal package in the AMS-LaTeX distribution. It adapts for use in
% LaTeX most of the mathematical features found in AMS-TeX; it is highly
% recommended as an adjunct to serious mathematical typesetting in LaTeX.
%
% When amsmath is loaded, AMS-LaTeX packages amsbsy (for bold symbols), amsopn
% (for operator names) and amstext (for text embedded in mathematics) are also
% loaded.
%
% amsmath is part of the LaTeX required distribution; however, several
% contributed packages add still further to its appeal; examples are empheq,
% which provides functions for decorating and highlighting mathematics, and
% ntheorem, for specifying theorem (and similar) definitions. 
%
% https://www.ctan.org/pkg/amsmath
\usepackage{amsmath}

% Allow page breaks in environments
\allowdisplaybreaks

% Mathtools provides a series of packages designed to enhance the appearance of
% documents containing a lot of mathematics. The main backbone is amsmath, so
% those unfamiliar with this required part of the LaTeX system will probably not
% find the packages very useful.
%
% Mathtools provides many useful tools for mathematical typesetting. It is based
% on amsmath and fixes various deficiencies of amsmath and standard LaTeX. It
% provides:
%
%  - Extensible symbols, such as brackets, arrows, harpoons, etc.;
%  - Various symbols such as \coloneqq (:=);
%  - Easy creation of new tag forms;
%  - Showing equation numbers only for referenced equations;
%  - Extensible arrows, harpoons and hookarrows;
%  - Starred versions of the amsmath matrix environments for specifying the
%    column alignment;
%  - More building blocks: multlined, cases-like environments, new gathered
%    environments;
%  - Maths versions of \makebox, \llap, \rlap etc.;
%  - Cramped math styles; and more...
%
% https://www.ctan.org/pkg/mathtools
\usepackage{mathtools}

% An extended implementation of the array and tabular environments which extends
% the options for column formats, and provides "programmable" format
% specifications.
%
% https://www.ctan.org/pkg/array
\usepackage{array}

% The package typesets fractions "nicely" — in the form 'a/b' (i.e., staggered
% with a slash between them, rather than directly one over the other).
%
% https://www.ctan.org/pkg/nicefrac
\usepackage{nicefrac}

% The package offers enhancements for theorem-like environments:
%
% -  easier control of layout; proper placement of endmarks even when the
%    environment ends with \end{enumerate} or \end{displaymath} (including
%    support for amsmath displayed-equation environments); and
% -  support for making a list of theorems, analagous to \listoffigures.
%
% https://www.ctan.org/pkg/amsmath
\usepackage[amsmath, thmmarks, hyperref]{ntheorem}

\theoremstyle{plain}
\theoremheaderfont{\normalfont\bfseries}
\theorembodyfont{\itshape}
\theoremseparator{:}
\theorempreskip{\topsep}
\theorempostskip{\topsep}
\theoremindent 0cm
\theoremnumbering{arabic}
\theoremsymbol{\ensuremath{\square}}
\newtheorem*{Lemma}{Lemma}

\theoremstyle{plain}
\theoremheaderfont{\normalfont\bfseries}
\theorembodyfont{\upshape}
\theoremseparator{:}
\theorempreskip{\topsep}
\theorempostskip{\topsep}
\theoremindent 0cm
\theoremnumbering{arabic}
\theoremsymbol{}
\newtheorem{Aufgabe}{Aufgabe}
\newtheorem*{Loesung}{Lösung}
\newtheorem*{Definition}{Definition}

\theoremstyle{plain}
\theoremheaderfont{\normalfont\bfseries}
\theorembodyfont{\upshape}
\theoremseparator{:}
\theorempreskip{\topsep}
\theorempostskip{\topsep}
\theoremindent 0cm
\theoremnumbering{arabic}
\theoremsymbol{\ensuremath{\square}}
\newtheorem*{Proof}{Proof}


%---- Sheets -------------------------------------------------------------------

% Header
\newcommand{\inserthead}{
\begin{center}
	\textbf{\course{}} \\
	\emph{\sheet{}}
\end{center}

% Authors
\noindent\rule{\textwidth}{1pt}
\small{\authors{}}
}

% First task
\setcounter{section}{\firsttask{}-1}
